% Chapter 1

\chapter{Study Context} % Main chapter title

\label{contextchapter} % For referencing the chapter elsewhere, use \ref{Chapter1} 

\lhead{Chapter 1. \emph{Study Context}} % This is for the header on each page - perhaps a shortened title

%----------------------------------------------------------------------------------------
\section{Obesity}
This section describes obesity from a clinical point of view to show the link between obesity and lifestyle. Obesity is the  result of a positive imbalance between what is consumed and what is expended by the body, where excess energy is stored in fat cells~\citep{steyn2006chronic}. This positive imbalance is due to two factors: (1) overeating, especially of an energy-dense diet (food that is high in either fat or sugar), and (2) a sedentary lifestyle. Results of well-conducted randomized control trials have concluded that these two factors increase the risk of obesity~\citep{swinburn2004diet}.

An energy-dense diet is one that is high in simple carbohydrates, which may result in a sharp elevation  of postprandial insulin levels that could lead to increased triglyceride storage in the adipose tissue depots. After a spike in the insulin level, the body senses that it has consumed all this energy but it doesn't need it all; hence it stores it in fat cells. If many cycles of storage occur, there will be an increase in fat depots, and if a person expends less energy than what is taken in, then the fat will remain in those depots. As insulin spikes happen, it is likely for a person to feel hungry soon after eating. This is due to an immediate conversion of all the sugar into energy, which the body then converts into fat upon realizing it exceeds the current required energy. The fat is then stored in depots; hence there is no sugar left in the bloodstream, and as a result the person may be tempted to keep on eating to compensate for depletion of glucose~\citep{bouchard1993exercise}. This poses a risk of going into many cycles of eating and being predisposed to binge-eating disorder~\citep{collins2009behavioral}. Individuals with binge-eating disorders lose self-control of their eating patterns. Therefore, as the person becomes obese, they are likely to lose control of their eating pattern and this may worsen their current situation of obesity. 

Most of the time, clinical diagnosis of obesity relies on measuring body mass index (BMI), which is obtained by dividing the person’s weight in kilograms ($kg$) by their height in metres squared ($m^2$). In some populations, a person is considered obese if their BMI is above 30 $kg/m^2$~\citep{steyn2006chronic} while in other populations the cut-off point may be different. There is controversy about using BMI alone, as some people can weigh more but may not necessarily be obese (having extra fat), because the extra weight may be due to having extra muscle, bone or water; hence in addition to BMI, it is recommended to also measure waist circumference to clinically diagnose obesity~\citep{janssen2004waist}. 

People with a BMI over 30 $kg/m^2$ are predisposed to the risk of co-morbidities related to obesity~\citep{de2000clinical}. Therefore, lifestyle modification is crucial in dealing with the obesity pandemic. The next section provides more background on the relevance of the problem within a South African context.
\section{Context Description}
This research conducted its studies with participants from low socioeconomic neighbourhoods of Cape Town, South Africa. There were four study sites, of which one was a diabetic and endocrinology clinic which is frequented by patients from low socioeconomic areas, while the remaining sites were three low socioeconomic townships in South Africa. 

The rationale for  deciding to work with participants from low socioeconomic neighbourhoods is supported by literature. A review by~\cite{dinsa2012obesity} suggests that in countries with medium human development index, of which South Africa is one, groups of low socioeconomic status are also affected by obesity, and the trend shows that women of low socioeconomic status are mostly affected compared with women of high socioeconomic status. Some barriers to adopting a healthy lifestyle that are present in low socioeconomic communities in the west also appear to occur in low socioeconomic urban communities in Cape Town. Studies that have been conducted in developed countries reveal that, in low socioeconomic areas, some environmental factors may influence behaviour patterns that predispose individuals to obesity. The environment may play a role in both promoting the intake of unhealthy food and in discouraging physical activity. Some of these factors could be lack of access to recreational facilities, a poorly designed built environment which lacks roads for pedestrians, or a lack of public transport that then promotes the use of private transport. The environments in which people live are complex and their individual and combined elements have a marked effect on behaviour and dietary intake~\citep{swinburn2004diet}. Food choices can be largely influenced by cultural issues and other factors such as price, portion size, taste, variety and accessibility of foods~\citep{ali2009factors}. The environment may also promote obesity by increasing the likelihood of consuming big portions of meals that are considered high in fat~\citep{hill1998environmental}. These contextual factors that may put individuals at risk of becoming overweight or clinically obese were also present in the context of participants of this research. Many low-income neighbourhoods in Cape Town are not safe, and this prevents people from doing simple physical activity such as walking. In addition, the meal outlets in townships sell food that is high in calories. In the contextual enquiry (Chapter \ref{contextualenqchapter}), the majority of the diabetic and obese participants claimed that healthy food in supermarkets is expensive, and they have to eat what the rest of the family eats because they cannot afford to prepare two separate meals. The preliminary study (Chapter \ref{contextualenqchapter}) observed that the notion of healthy food is not quite understood, therefore the application that was tested in this context helped participants to understand that you can still live a healthy life by utilizing whatever resources you have. The aim of this research was to explore how to design in order to support motivation in intermediated use of personal health informatics in the context of South African low-income townships. 

The problem of obesity in South Africa is quite alarming. Statistics show that almost 60\% of South Africans are overweight~\citep{ng:global}. Urbanization and emigration of people from upcountry to cities  have been suggested as possible reasons for the adoption of unhealthy behaviours, as the city lifestyle encourages people to be more sedentary and increase their consumption of caloric-dense food~\citep{ali2009factors}. The populations that live in low socioeconomic areas face numerous challenges. Most apartheid policies regarding health did not focus on these populations, and some of the current health and economical concerns are a result of amplifications of apartheid social clusters \citep{benatar2013challenges}. This is why it is crucial to focus on low socioeconomic areas in order to understand technology design constraints for these marginalized populations. In addition to the health concerns, we have already discussed how sharing of technology and the presence of indirect use could hamper utilization of technology for health interventions in low socioeconomic areas of developing countries.   
\begin{flushright}
\end{flushright}
