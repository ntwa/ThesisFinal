% Chapter 1

\chapter{Conclusions and Future Research} % Main chapter title

\label{discussionchapter} % For referencing the chapter elsewhere, use \ref{Chapter1} 

\lhead{Chapter \ref{discussionchapter}. \emph{Conclusions and Future Research}} % This is for the header on each page - perhaps a shortened title

%----------------------------------------------------------------------------------------
The main research questions were centred around the factors that could affect utilization of a personal health informatics application through intermediaries, and also the effectiveness of gamification in increasing both engagement of intermediaries and collaboration between members of a pair in intermediated use. This chapter revisits the research questions and discusses how they were addressed. It also summarizes takeaways which can be regarded as design considerations for motivational affordances in intermediated use of a self-monitoring application that promotes healthy behaviours. These design considerations reveal social factors that can contribute to the success of an intervention such as the one in this study, and motivational strategies that can be utilized to keep both intermediaries and beneficiaries engaged with a personal health informatics (self-monitoring) system/application.

\section{Discussion of Research Questions}
The main focus of this research was to uncover how social factors and persuasive systems' inspired motivational affordances might impact intermediated use of personal health self-monitoring applications. Intermediated technology use is an interaction model that is prevalent in low-income areas of the developing world. Many health self-monitoring applications have motivational affordances that cater for personal or direct usage, but such motivational affordances have not been explored in the context of intermediated use. In order to understand design requirements for motivational affordances targeting intermediated use, this research aimed to provide answers to the following research questions.

\textbf{RQ1}: What is the role of social-technical settings in the intermediated use of a gamified self-monitoring application targeting the promotion of healthy eating and physical activity? 

This research question had two sub-questions. The first aimed to identify prerequisite factors that could affect intermediated use, and the second explored the extent to which an understanding of the identified factors is important in the context of intermediated use. In order to provide answers to these research sub-questions, I conducted a series of studies. These studies included one contextual enquiry (Chapter \ref{contextualenqchapter}) and two consecutive evaluations of two versions of the prototype (Chapters \ref{prototype1chapter} and \ref{prototype2chapter}). 

The most noteworthy factors that came to light in the aforementioned studies included but were not limited to, social relationship; collaborative reflection using a shared device; and motivational affordances, either as part of the app or socially constructed as a result of using the app. 

Informative evaluations suggested that involving children who are family members is the key to the success of this kind of intervention, and running an app running on a shared device increases the tendency of members of a pair to reflect collaboratively. In such contexts, the interaction is no longer just a matter of help-seeking and help-giving, but more a collaborative effort towards a joint goal. The use of collaborative interfaces for organizing health information within family settings has been extensively studied in computer supported collaborative work (CSCW) literature. \cite{colineau2011motivating} designed a system that supported a family to select a collective health goal and receive feedback that entailed comparisons between families. Their system was found to encourage members from within a family, or members of different families, to work together and, in particular, to help each other in finding ways to live a healthy lifestyle. 

Family settings can provide an ideal opportunity for members to discuss health issues collaboratively. For instance, the three chapters on the evaluations of the prototype (Chapters \ref{prototype1chapter}\thinspace-\thinspace\ref{summativeevalchapter}) indicated that collaboration between members of a pair had positively impacted intermediaries with knowledge about eating healthily, although that knowledge targeted beneficiaries. In addition, intermediaries in some cases logged their data about meals because what they ate was not different from what had been eaten by their respective beneficiaries. A study by~\cite{grimes2009toward} identified four key areas in which sharing of, and reflection on, health information can be leveraged within family context: (1) overlaps of routines between family members through shared meals and space, can provide opportunities for collaborative data logging and reflection among family members; (2) sharing is done at the expense of trying to balance competing values of openness, caring, and modelling with the value of protection; (3) Sensitivity regarding comparisons and competition based on health information in the context of the family, as these may have negative consequences; and (4) collaborative sharing of and reflecting on health information can foster a family's bond. In the context of this research, it was evident that the app had increased the bond between participating family members, as the majority of them claimed that they were interacting more often. This is also demonstrated by playful behaviours that were exhibited in the process of sharing information, as it shown in an excerpt shown below from Table \ref{table:collaborativereflection} in Chapter \ref{prototype2chapter}:

\userquote{\textbf{Zandiwe}, a beneficiary} {` When she [\textbf{Lindiwe}, a girl aged 16 yrs] got time, when she is done with her homework she comes and sees the app. And then laughs at me like `Yo yo yo [an interjection for Xhosa speakers to express the feeling of amazement by something], you can walk yo yo yo', like `you walked a lot today' and what what [referring to other things said by Lindiwe].''}

In existing literature on computer supported collaborative work, the emphasis is on parents trying to model the health behaviours of their children, as it was emphasized in a study by ~\cite{saksono2015spaceship}, where a collaborative exergame was developed with the goal to help the kids learn from their parents, even though the collaborative environment was beneficial to both parents and children.  

In this research, it was the other way around, as children were attempting to nudge (model) their parents to live healthily -- it was not only a parent attempting to guide their children, as suggested by literature. The children also became guides to their parents about healthy choices. This was mediated by an existing familial relationship. In addition, in most cases where pairs consisted of a parent working with a child, the intermediary tended to be reasonable, when asked by their beneficiary users to fulfil requests for translation of intents to interact with the app, even when intermediaries felt their autonomy was being violated. Empathy led to such intermediaries becoming accountable for the well-being of the people they cared about. The bond was further strengthened by the presence of motivational affordances, which had a role to play in making intermediaries believe that information on the app was theirs as well and not only for the beneficiary users who were being assisted. As a result, those intermediaries became responsible team players. Nudging and cooperation in this context can be viewed as a form of social support that beneficiaries received from within their families. Through working together, reflection was done collaboratively and not at personal level as is common in existing personal health informatics applications.  However, perceived interest for motivational affordances differed between intermediary users and beneficiary users. Comparison based on abstract things like points was less meaningful to older beneficiaries, as they tended to value perceived benefits and social support from others more. In the absence of interaction among beneficiaries, there was a tendency to have less engagement from the side of beneficiaries. Hence strategies that are applied for this user group need to take into consideration the availability of social support from people who already know each other (other adults or beneficiaries). This kind of social support tended to increase relatedness among beneficiary users, as reported on in the evaluation of the prototype in Chapter \ref{prototype2chapter}. One important conclusion from this finding is that there is a need to have separate persuasive strategies for beneficiaries and intermediaries. Existing game mechanics may work well with intermediaries, while for beneficiaries social support should be encouraged in order to leverage the motivational affordance provided by social comparison.

In response to the first main research question, it can be concluded that motivational affordances are able to foster collaboration and subsequently a relational bond between members of a participating pair in an intervention, provided that there is a prior social relationship between members of a pair. This implies that the combination of motivational affordances and familiar relationships is crucial in making a collaboration more interesting and enjoyable. Therefore, a prior social relationship and perceived motivational affordances were the main determinants for two users from two sets (intermediary set and beneficiary set) to view any efforts in interaction as being carried out on behalf of the team and not the beneficiary user alone.  

The second research question is provided below. This aimed to explore the impact of using gamification as a means to motivate collaboration that leads to the intermediated use of a self-monitoring application.

\textbf{RQ2}: How does gamification play a role in motivating the intermediated use of a self-monitoring application that targets the promotion of healthy eating and physical activity?

This research question was broken down into seven research sub-questions, as provided below:

\begin{enumerate}[label=\alph*.]
\item What is the impact of gamification on supporting the self-determination of intermediary users to engage with a self-monitoring application in intermediated use?
\item What is the impact of gamification on supporting the self-determination of beneficiary users to engage with a self-monitoring application in intermediated use?
\item What is the impact of gamification on the frequency of utilizing the self-monitoring application in intermediated use?
\item What is the impact of gamification on the motivation of beneficiaries to self-monitor diet?
\item What is the impact of gamification on the motivation of beneficiaries to self-monitor physical activity?
\item How does the presence of gamification affect the relationship between the two members of a pair participating in intermediated use?
\item To what extent might gamification encourage or discourage internalization of intermediated use behaviour?
\end{enumerate}

In order to address these sub-questions, I did conduct a summative evaluation (Chapter~\ref{summativeevalchapter}) that compared between a system without gamification and a system with gamification. 

For research sub-question 2(a), gamification demonstrated the potential to increase the perceived competence of intermediary users in using the self-monitoring app with their respective beneficiary users, while aspects of perceived autonomy and perceived relatedness did not show improvement. I have highlighted many insights on design considerations concerning autonomy. The most important was users' freedom to choose which gamification features to participate in and at what level, in order to cater for intermediary users with different personality types and skills. Concerning relatedness, features that promoted socialization and relatedness were only effective for users who were not co-located, and this resonated with findings from other literature. 

For research sub-questions, 2(b), 2(c) and 2(d), the perceived competence, the perceived autonomy and the perceived autonomy (to use the app, to self-monitor diet and to self-monitor physical activity) were the same for each respective comparison between logbook and gamification conditions. However, the presence of a self-monitoring app, regardless of whether or not the app had gamified motivational affordances or not, appeared to increase the self-determination of beneficiaries to monitor their health at endline in comparison to baseline. A significant change was observed in self-monitoring of diet, since there was less prior interest or knowledge in tracking of diet, while for physical activity participants seemed to already be doing implicit tracking through approximation of amount of physical activity done while doing daily errands. For the majority of the beneficiary participants, gamification was of less importance for reasons that have already been highlighted in the discussion of the first research question. Reflecting on the series of user evaluation studies that I conducted, insights reveal that the gamification motivational affordances utilized in this study may work more effectively with intermediaries (who in this case were mostly children) and young beneficiaries. 

For the last three sub-questions, 2(e), 2(f) and 2(g), gamification increased frequency of usage through intermediaries, and fostered collaboration in cases where both intermediaries and beneficiaries were interested in gamification features. However, this usage and collaboration, as reported on in Chapter \ref{summativeevalchapter}, appeared to be mostly due to introjected internalization, and had negative consequences. Literature suggests that despite the ability of gamification to add user experience to an uninteresting activity, there are negative consequences as a result of increased competition~\citep{jia2016personality}. It is even more concerning when these negative consequences occur in health settings~\citep{grimes2009toward}. To curb the negative effects of excessive social comparison and competition, literature emphasizes the need to support challenges at the level of a task-mastery climate, which has been found to promote intrinsic motivation, unlike competition that has an ego-involved climate, which has a tendency to harm intrinsic motivation~\citep{saksono2015spaceship}. In situations where ego is involved, regulation tends to be done for the sake of maintaining one's self-worth, and is considered to be of type introjected~\citep{ryan2000:self}. In cases of introjected regulation, the behaviour that is being promoted may be obscured; hence it will not be perceived as matching the core values and beliefs of an individual. Having a leaderboard in this intervention was counterproductive. Introjected regulation was more evident in Chapter \ref{summativeevalchapter} (Summative Evaluation) than when the prototype was evaluated in Chapters \ref{prototype1chapter} (Prototype I) and \ref{prototype2chapter} (Prototype II). In the qualitative feedback recorded in Chapter \ref{summativeevalchapter}, leaderboard alone was discussed; participants had very little to contribute when features such as fish tank or botanical garden were mentioned during the conversation. This was as opposed to Chapters \ref{prototype1chapter} and \ref{prototype2chapter}, where the interviewees discussed in detail about those features, implying that participants' engagement with those features was meaningful. What I observed was that the number of active intermediary users had increased in the summative evaluation (Chapter \ref{summativeevalchapter}) and was distributed evenly across the leaderboard hence negative effects as a result of extreme competition became conspicuous, while in the chapters covering the informative evaluations, intermediary users who were active were all at the top of the leaderboard and that did not focus much on outperforming others. The excerpt below, from the evaluation of the prototype in Chapter \ref{prototype2chapter} (Prototype II), shows an intermediary user who was not particularly concerned about the competition, as users/teams that were at the bottom of the leaderboard did not pose any challenge to her (most intermediary users at the bottom were those who were less active on usage because of less interaction with their respective beneficiary users). The competition was mostly between three intermediary users who were at the top.  

\userquote{\textbf{Lulama}, an intermediary} {``I am concentrating on winning. I do look at other people but I am like, I am going to beat this one. So I don't need to look at them. Because I saw that the fact that we are on top three, I can do this. I can compete with others and win because we are in the top three. The ones that I am in top three with, yah [I compete with them]. Because I don't want be the third or second. I have to be the first. The others I know I have already passed them.''}

Building on that observation, giving feedback on only the top users may be a good design consideration. During this study, the system broadcasted text messages (SMSes) daily, informing all pairs about the top three teams/pairs on the leaderboard. This appeared not to negatively affect the motivation of intermediary users who had not visited the leaderboard and were not in the top three. Displaying the whole leaderboard may not be a good idea, as it can create pressure that leads to introjected regulation. Features such as fish tank and botanical garden, which appeared to promote task-mastery, could be useful in engaging beneficiary users as well. For instance, in one scenario reported on in Chapter \ref{prototype2chapter} (Prototype II),  a beneficiary user was dissatisfied with the look of her garden (team's garden). 

\userquote{\textbf{Lulama}, an intermediary} {``She [Nokhanyo] saw the garden. The first day she saw just the house and brownish. She is like `what is this'. I told her. She said `Aha! [expressing dissatisfaction]. It must look green and healthy'. And then she saw the garden again and said `It is looking good'[This participant is explaining how the botanical garden challenged her mother to do better].''} 

Features that promote task mastery should be more visible. If designers have to use a leaderboard, they need to be cautious of possible negative impacts on users' perceived enjoyment, despite a leaderboard's ability to foster relatedness~\citep{sailer2013:psychological}. In cases where social comparison is extreme, a leaderboard can damage the relationship between an intermediary user and a beneficiary user because intermediaries feel let down by their beneficiaries. Such a scenario was exhibited in the following excerpt presented in Chapter \ref{summativeevalchapter} (Summative Evaluation)
 
\userquote{\textbf{Jenner}, a beneficiary} {``Sometimes maybe I forget to take the phone when I go walking and he would ask me `Did you take the phone with you?' `Ooh Gosh I forgot.'  Because when I walk to Park Town to exercise, sometimes  I am in such a hurry I forget the phone, and he gets crossed with me.''} 

In this case, the intermediary got angry because of his mother's tendency to forget the pedometer (phone) when she went walking. This highlights the importance of paying more attention to features that promote a task mastery climate.

In the results reported on in Chapter \ref{summativeevalchapter} (Summative Evaluation), the negative impact of social comparison was more apparent in intermediaries. Beneficiaries who participated in informative evaluations (Chapters \ref{prototype1chapter} and \ref{prototype2chapter}) also did a lot of social comparison, but their motivations seemed not to be affected in a negative -- in most cases it proved to increase their enjoyment and motivation to engage with the app. Social comparison challenged beneficiaries to continuously set and revise goals for healthy living. Therefore, for beneficiaries, it showed indications of promoting a task-mastery climate.  In addition, Chapter \ref{prototype2chapter} (Prototype II) showed that beneficiaries were engaged differently outside gamification context, but were able to collaborate with their intermediaries. Intermediary users might have goals of achieving rewards through game mechanics, while beneficiary users' goals can be to accumulate more steps or to eat healthily and receive social support from among themselves. Even though the goals of the two sets of users are entirely different, as long as motivational needs of each set of users are met, it can increase chances of collaboration between an intermediary and a beneficiary. 

In conclusion, it is clear that both gamification and other motivational affordances that were indirectly situated in the app promoted collaboration between participating pairs that had prior social rapport. In addition, gamification -- in particular, social comparison and competition, either socially constructed by users or implemented as an intentional design goal, increased the engagement of both intermediary and beneficiary users.
Gamification was effective in increasing the engagement of intermediaries, but it had some challenges and limitations that need to be addressed, as it appeared to affect both intrinsic motivation and internalization in the summative evaluation (Chapter \ref{summativeevalchapter}).
 
\section{Limitations}
The aim of this study was to uncover social technical settings affecting the intermediated use of gamified personal health systems. There are two main limitations with this study. First, the researcher (myself) relied on interviews to capture user experiences, as a result, findings were generated based on limited interactions with participants. An ethnographic study and a diary might have been useful in capturing profound insights on user experience that could not be recalled or revealed in interviews.~The second limitation of this study is that the evaluation of motivational affordances was done in a holistic manner, making it difficult to discern the impact on individual users. In addition, the evaluation of motivational affordances did not take into consideration how the personality of a user affected the intervention. There was a limitation on understanding the effect of different motivational affordances, as the researcher (myself) used several motivational affordances (leaderboard, fish tank, badges and botanical garden), hence it was difficult to discern the isolated effect of each one of them.   

Despite these limitations, the series of these small studies done in this research laid a good foundation for understanding the social dynamics that play a role in the intermediated use of personal health informatics applications. This foundation is also important in formulating several hypotheses that can be tested in larger empirical studies. 

\section{Future Directions}
In order to enhance user experience, support for factors such as task mastery, reflection, enhancement of collaboration within a family (intra-family), and inter-family collaboration should be further explored. In addition, factors such as personality of users, and different styles of parenting such as neglectful, permissive and authoritarian, should be explored within the realm of intermediated use of a personal health informatics application. How users and technology are arranged is also an important area that needs further exploration, as it may affect the achievement of optimal flow. This issue had an effect on the flow of both sets of users in this research. 

Another concept that needs to be further explored by future studies is sustained usage over a long period of time. Long-term usage and prolonged benefits of personalized apps are still debatable. One study conducted a two-year trial that compared three strategies: an interactive smartphone application on a cellphone (CP); personal coaching enhanced by smartphone self-monitoring (PC); and handouts (pamphlets) as the control group. The study found the smartphone strategies to be better than the control in the short term, but long term, the results were not different~\citep{svetkey2015cell}. In their conclusion, they argued that usage of these smartphone apps should go hand in hand with other conventional strategies. In addition, long-term engagement is a key challenge in personalized apps for health. Once the novelty effect has worn off, users may often revert to their old habits. Gamification is not exceptional when it comes to the issue of the novelty effect. When technology is utilized through intermediary users, it can be even more challenging to sustain engagement, since we are dealing with more than one layer of users. This is still a grey area that needs further exploration. However, the concept of sustained usage can be debatable, based on the following argument. We have seen in the literature review section that, users of personal informatics are expected to transition back and forth between the discovery phase and the maintenance phase~\citep{li2011understanding}. While in the maintenance phase, it is possible for usage to stop once the user has reached the last level of trans-theoretical model (TTM)~\citep{grimley1994transtheoretical} -- in this level the user has already established a healthy living routine that can be followed through with determination, hence it is possible for them to stop using their personal informatics system, as was observed in a study by~\cite{lin2006:fish}. Therefore, sustained usage should be interpreted as a situation where beneficiaries have intents to interact with personal informatics systems and intermediaries are available to help. Sustained usage should be explored in that dimension.    

I also recommend that future studies should explore the impact of each motivation affordance in isolation instead of combining them under one interface. This should extend the discussion of features that harm intrinsic motivation and those that promote.

The last concept that needs exploration is whether one could use the same model to support older adults in higher income areas. There could be some similarities between lessons learned through this study and ones that may be uncovered by repeating the same research in higher-income areas.
\begin{flushright}
\end{flushright}
