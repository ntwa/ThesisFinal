% Chapter 1

\chapter{Summative Evaluation} % Main chapter title

\label{summativeevalchapter} % For referencing the chapter elsewhere, use \ref{Chapter1} 

\lhead{Chapter 1. \emph{Research Findings}} % This is for the header on each page - perhaps a shortened title

%----------------------------------------------------------------------------------------
\section{Recruitment of Participants}
With help of a research assistant who was a resident of Langa,  we managed to recruit a total of fourteen adult participants (beneficiary users). We recruited these participants from two Cape Town's townships: Langa, and Athlone. In Langa there were six adult participants while in Athlone there were nine adult participants. The average age of these adult participants was 44.21 years with a standard deviation (S.D) of 9.99 years. The youngest adult was 26 years of age while the oldest was 60 years of age. Thirteen participants were females. 

Each adult participant elected one of their children/grand children to become their intermediary user hence form a pair. Each pair was required to work together in managing the wellness of one member of a pair (a beneficiary user). All beneficiary users were working with their children but one whom was was working with her grand child.  The average age of children participants (intermediary users) was 15.42 (S.D=2.06) years. The youngest intermediary user was 12 years of age while the oldest was 20 years of age. The number of females and males intermediary users were equal. 

I gave detailed information of what the study was all about to both intermediary and beneficiary participants. I informed them about different modes of which I will collect data. All beneficiary participants signed informed consent forms to agree to be part of the study. Since all intermediaries were under 21 years of age, they signed assent forms which were also signed by their parents/guardians who were part of the study.

I allocated one day to teach intermediary participants on how to use the "Family Wellness App". In addition, each intermediary was given a user manual. After the training, I gave out one Android phone Samsung(GT-S5300) to each pair of participants. These phones were installed with tow natives apps. The first one was a pedometer and the second one was the main "Family Wellness App". The "Family Wellness App" loaded all its content from a web application hosted remotely. The app was used for a total period of six weeks. Each pair of participants provided the service provider's number of their SIM Card that will be inserted on their given Android phone. I allocated 1.3 GB of data to each SIM CARD. In addition each beneficiary participant was given a total of ZAR 240 as a compensation for transport and time throughout the duration of the study. The details of the experiments are outlined on the next section.

\section{Experiments}
This phase of the study evaluated the effectiveness of gamification/rewards in motivating both intermediaries and beneficiaries to engage with the family wellness application (Family Wellness App). I compared two versions of the applications. The first version of the application was simply a logbook or journal that allows a pair of users to record and view wellness data of a beneficiary member of the pair. With the logbook app users could view physical activity feedbacks and recording and viewing summaries of nutrition components of food consumed by a beneficiary within a pair. The second version of the application was an extension of logbook to include rewards/gamified subsystem.I carried this evaluation for a period of six weeks. I carried out the experiments from the mid October 2015 to the end of November 2015.  The details of how experiments were designed and how data were collected are presented on the next sub-sections.
\subsection{Experiment Design}
I used ``within-group'' design for my experiments. In within-group design, the same group of participants is exposed to different experimental conditions. This helps to minimize the number of group needed to test hypotheses as only one group is used for control and intervention. Another advantage of within-group design is that it minimizes the effect of confounding factors. The only problem with this approach is the learning effect and it lengthens the duration of the study. In order to minimize the impact of the learning effect on the outcome, I randomly assigned pairs of participants to two groups referred to as experimental sequences. The first experimental sequence started with the ``Logbook App''  and finished with the ``Gamified App''. The second experimental sequence started with the ``Gamified App'' and finished with the ``Logbook App''. I used the following abbreviations ``LG'' and ``GL'' to refer to the first and second experimental sequences respectively.

A total of seven pairs of participants were assigned to the LG group while the remaining seven pairs were assigned to the GL group. Both groups spent the first four weeks in their first experimental conditions of which the ``Logbook App'' for the LG group and the ``Gamified App'' for the GL group. After 27 days each group was switched to a different experiment condition. The LG group started using the Gamified App while the GL group started using the Logbook App. The second phase of the experiment lasted for a total of 14 days. In the next sub-section, I provide details of how data were collected during the duration of 41 days (6 weeks) of running the experiments 

\subsection{Data Collection Methods}
Data collection was a triangulation of application's logs,questionnaires and interviews. 
\subsubsection{Family Wellness App Logs}
Application's Logs consisted of information regarding the time there were users activities on the app, the pair that was accessing the app, and the functionality that was being accessed by that  pair. Logs were categorized to their respective experimental condition. 
\subsubsection{Questionnaires}
I administered questionnaires at the baseline, mid-line (during switching of experimental conditions), and end-line. The list of questionnaires is provided below.

\textbf{Baseline Questionnaires}

At baseline both intermediary and beneficiary participants filled their respective questionnaires. 
\begin{itemize}
\item{\textbf{Intermediaries}}: Intermediaries participants' baseline questionnaire had three sections. The first section captured demographic information such as age,gender, and number services/apps used on cellphones. The second section included an IMI (Intrinsic Motivation Inventory) questionnaire  to assess participants' intrinsic motivation in using cellphones. The third section included an IMI questionnaire to assess participants' intrinsic motivation in helping their parents with cellphone based tasks.
\item{\textbf{Beneficiaries}}: Beneficiary participants' baseline questionnaire had four sections. The first section captured demographic information such as age,gender, and number services/apps used on cellphones. The second section included an IMI questionnaire to assess participants' intrinsic motivation in using cellphones. The third section included an IMI questionnaire to assess participants' intrinsic motivation in self-monitoring of diet/nutrition.The third section included an IMI questionnaire to assess participants' intrinsic motivation in self-monitoring of physical activity.
\end{itemize}

\textbf{Midline Questionnaires}
Also at midline both intermediary and beneficiary participants filled their respective questionnaires. 
\begin{itemize}
\item{\textbf{Intermediaries}}: Intermediaries participants' midline questionnaire had only one section which included an IMI questionnaire  to assess participants' intrinsic motivation in using the family wellness app.
\item{\textbf{Beneficiaries}}: Beneficiary participants' baseline questionnaire had four sections. The first section included an IMI questionnaire  to assess participants' intrinsic motivation in using the family wellness app. The third section included an IMI questionnaire to assess participants' intrinsic motivation in self-monitoring of diet/nutrition.The third section included an IMI questionnaire to assess participants' intrinsic motivation in self-monitoring of physical activity.
\end{itemize}

\textbf{Endline Questionnaires}
At endline both intermediary and beneficiary participants filled their respective questionnaires. 
\begin{itemize}
\item{\textbf{Intermediaries}}: Intermediaries participants' endline questionnaire had only one section which included an IMI questionnaire  to assess participants' intrinsic motivation in using the family wellness app.
\item{\textbf{Beneficiaries}}: Beneficiary participants' endline questionnaire had three sections. The first section included an IMI questionnaire  to assess participants' intrinsic motivation in using the family wellness app. The third section included an IMI questionnaire to assess participants' intrinsic motivation in self-monitoring of diet/nutrition.The third section included an IMI questionnaire to assess participants' intrinsic motivation in self-monitoring of physical activity.
\end{itemize}
\subsubsection{Interviews}
I also conducted short unstructured interviews at midline and endline. I only picked fewer intermediaries and beneficiaries for the interviews. Interviews responses were important in supplementing data collected through questionnaires and application's logs.
\section{Findings}
There are two primary outcomes in analysing the findings. The first primary outcome is usage of both systems. The second primary outcome is intrinsic motivation of intermediaries in using the both systems.There are also secondary outcomes and these are: intrinsic motivations of beneficiaries in using the wellness application; self-monitoring of diet/nutrition; and self-monitoring of physical activity.
\subsection{Primary Outcomes}
The average number of days on which user used both versions of the application was 10.5 (SD=7.39) days. The most active usage was from a pair that utilized the app for a total of 26 days. The less active usage was from a pair that had used the app for only two days out of 41 days. 

I compared usage between a gamified system and logbook system by counting both the number of days and total number sessions between the two experimental conditions. Since the number of days and total sessions were relative to how many days users were exposed to a particular experimental condition, I divided the number of sessions with total days on which users were exposed to a particular experimental condition to get the ratio of days with respect to the total number of days and number of sessions per day. Therefore, comparison was based on a relative number and not an absolute. For instance, lets say pair one is from a LG group which they had access to a logbook system for 27 days and a gamified system for 14 days. Suppose the number of sessions in Logbook is 50 and the number of sessions in Gamified system is 30 then the relative number of sessions per day spent on logbook is 50 divide by 27 days and the relative number of session spent on  a gamified system is 30 divide by 27. 
 
Before comparing the two versions of the system on usage, I had to test if the differences of both the ratio of days and number of sessions per day between the two systems follow a normal distribution so that I can identify a proper statistical test to use. The difference between logbook and gamification didn't have a normal distribution in both the ratio of days and number of sessions per day. I applied a natural logarithm transformation on the data and the difference of logs of both cases followed a normal distribution. To test for a normal distribution I used ``Shapiro-Wilk Normality Test''\footnote{http://sdittami.altervista.org/shapirotest/ShapiroTest.html}.
\newline 
\begin{table}[h!]
  \begin{center}
    \caption{Usage Comparison between Logbook and Gamified Systems for 14 pairs of users}
    \label{table:usagewellness1}
	\begin{tabular}{|c|c|c|}
		\hline
		Log mean &Logbook App&Gamified App\\
		\hline
		 \multirow{2}{*}{Ratio of days}&M=0.16;SD=0.15&M=0.24 ;SD=0.18\\\cline{2-3} 

		 &\multicolumn{2}{|l|}{t(13)= 1.4927 ; p=0.1594 ; 95\% CI= -0.184 to 0.034} \\
\hline
   		 \multirow{2}{*}{ Number of sessions/day}&M=0.22 ;SD=0.22&M=0.36;SD=0.33\\\cline{2-3} 
		
		 &\multicolumn{2}{|l|}{t(13)=1.5519 ; p=0.1447 ; 95\% CI= -0.341 to 0.056} \\
\hline
	\end{tabular}
  \end{center}
\end{table}
\newline  
 
I carried out a student t-test on logarithmic transformed data.  On comparison of usage between a gamified system and logbook system, the former had a higher mean log of ratio of days  and mean log of number of sessions per day compared to the latter without statistical significance.
The results are shown on Table \ref{table:usagewellness1} above.

However there was a peculiar phenomenon of usage pattern that didn't resonate with self-reported IMI in using Family Wellness App. There were almost six users who had a negative experience in interacting with the gamified system of which two had a significant drop in usage when switched from Logbook to Gamification and this contributed to not achieving a statistical significance in comparing the usage of the two systems as there was an anticipation for usage to be sustained switching to gamification. The negative experience of six users was reflected on their intrinsic motivation to use the family wellness app. A gamified system appeared to harm their existing intrinsic motivation as they scored less in IMI scores for a gamified system when compared to a logbook system except for one user. This suggests that instead of a gamified system to improve intrinsic motivation of these users, it harmed intrinsic motivation. I examined what was common among these six users. The five out of six users never managed to make any progress in attaining a single reward due to various reasons despite their efforts. Rewards were earned based on efforts made by beneficiary users. A formula for computations of badges and ranks on the score board was based on an average number of steps walked by a beneficiary user and number of days an intermediary had used the application. For each badge there was a threshold. The first two users tried to use the app more often but their beneficiaries were not walking enough number of steps hence they remained on the same position. Therefore, they didn't enjoy the gamified system. The other three users had technical problems, of whereby two of them had their pedometers stopped transmitting data hence they couldn't advance in badges while the remaining user experienced problems every time he attempted to access the app, therefore he couldn't gain any points on usage. The sixth user managed to advance in badges but also had a negative experience as the result of a multiplier effect since he was close to the two users who had their pedometers not transmitting data. She didn't appreciate her advancement in badges because she reckoned that her peers did more efforts than her but they were not getting anything so she didn't understand why she was ahead of them. Therefore she didn't enjoy the gamification and hence her intrinsic motivation in using the family wellness app was below compared to when she was using logbook. 

I repeated the above analysis on usage without the three pairs who had usability problems. Two of these pairs were from the ``LG'' group while one was from the ``GL'' group. The results of the analysis are presented on Table\ref{table:usagewellness2}. A student t test demonstrated that the means log of both  the ratio of days and number of sessions per day were significantly higher in the Gamified system compared to the Logbook system.
\newline 
\begin{table}[h!]
  \begin{center}
    \caption{Usage Comparison between Logbook and Gamified Systems for 11 pairs of users}
    \label{table:usagewellness2}
	\begin{tabular}{|c|c|c|}
		\hline
		Log mean &Logbook App&Gamified App\\
		\hline
		 \multirow{2}{*}{Ratio of days}&M=0.14;SD=0.14&M=0.27 ;SD=0.18\\\cline{2-3} 

		 &\multicolumn{2}{|l|}{t(10)= 2.5893 ; p=0.0270 ; 95\% CI= -0.246 to -0.0177} \\
\hline
   		 \multirow{2}{*}{ Number of sessions/day}&M=0.18 ;SD=0.20&M=0.43;SD=0.34\\\cline{2-3} 
		
		 &\multicolumn{2}{|l|}{t(10)= 2.7479 ; p= 0.0206 ; 95\% CI=  -0.442 to -0.046 } \\
\hline
	\end{tabular}
  \end{center}
\end{table}
\newline  
In Comparing the IMI to use the wellness app of intermediaries I separated between users with negative experience on using a gamified system and users with the same or experience greater than the one from Logbook.
 
\begin{flushright}

\end{flushright}
